\documentclass{article}
\usepackage{ctex}
\usepackage{amsmath}
\usepackage{graphicx}
\usepackage{subfigure}
\graphicspath{{figure/}}
\usepackage[section]{placeins}
\usepackage{float}
\usepackage{appendix}
\usepackage{amstext}
\usepackage{listings}
\usepackage{geometry}
\geometry{a4paper,scale=0.8}
\lstset{language=Matlab}%语言
\title{螺旋波的产生及混沌控制}
\date{2020.6.16}
\author{俞炀}

\begin{document}
	\maketitle
	\begin{abstract}
		文章以FitzHugh-Nagumo模型为基础,详细阐述了螺旋波及其湍流态的产生方法,并利用行波法对其进行了有效控制。	
		
		\textbf{关键词:}\qquad 斑图动力学  \qquad 螺旋波 \qquad 行波法 
	\end{abstract}
	\section{螺旋波模型简介}
		\subsection{数学表述}
		螺旋波通常产生于反应扩散系统。在本文中,我们用一个描述在可兴奋性媒介中波动传播的模型来产生,并控制螺旋波及其湍流态。此模型为FitzHugh-Nagumo模型,其数学表述如下:
		\begin{align}
			\label{eq1}
			\frac{\partial u}{\partial t}=-\frac{1}{\varepsilon}u(u-1)(u-\frac{v+b}{a})+\nabla^2u\\
			\frac{\partial v}{\partial t}=f(u)-v\\
		\end{align}
		
		其中:
		\begin{equation}
		f(u)=\left\{
		\begin{aligned}
		&0&,&0\leq u<\frac{1}{3} \nonumber\\
		&1-6.75u(u-1)^2&,&\frac{1}{3}\leq u \leq 1 \nonumber\\
		&1&,&u>1 \nonumber
		\end{aligned}
		\right.
		\end{equation}
		
		此模型给出的边界条件为零流边界条件,即在边界上满足:
		\begin{equation}
			\frac{\partial u}{\partial n} = 0
		\end{equation}
		
		方程\ref{eq1}中v的初值设置为0;u在左边界上的初值为1,其余均为0.
		
		其余参数罗列如下:a=0.84,b=0.07,$\varepsilon \in [0,0.07]$(非混沌状态)。
		
		\subsection{模型的Matlab求解与演示方法}
		本文中将对三个问题进行处理和讨论:
		\begin{itemize}
			\item 行波与靶波的产生
			\item 螺旋波的产生
			\item 混沌现象的产生
			\item 使用行波法实现螺旋波与混沌现象的控制
		\end{itemize}
		以上四个问题采用的方法为显式欧拉法,其中时间步长为0.05,空间格点间距选为0.5;
	
		对于前三个问题,本文在$800\times800$的格点上对该模型的不同条件进行求解并实现可视化效果。由于最后一步在尝试过程中发现$800\times 800$格点上行波对于混沌/螺旋波的控制过程所需时间过长,因此采用$100\times 100$格点进行实现。
	\section{行波与靶波的产生}
		当我们对于方程的左边界施加一个持续性的刺激,即令左边界初值周期性输出1时,我们可得到行波的图像,如图\ref{t1_1}所示;
		如需产生靶波图像,只需在所有u的初值中间四个点设置为1,其余u值为0即可。其具体图像如下图\ref{t1_2}所示:
		\begin{figure}[H]%H表示优先放置的图片位置
			\centering
			
			\subfigure[行波]{
				\begin{minipage}[t]{0.4\linewidth}
					\centering
					\includegraphics[width=2in]{t1_1.png}
					\label{t1_1}
					%\caption{fig1}
				\end{minipage}%
			}%
			\subfigure[靶波]{
				\begin{minipage}[t]{0.4\linewidth}
					\centering
					\includegraphics[width=2in]{t1_2.png}
					\label{t1_2}
					%\caption{fig2}
				\end{minipage}%
			}%	%这个回车键很重要 \quad也可以
			\centering
			\caption{在不同初值条件下产生的行波与靶波}
			\label{t1}
		\end{figure}
	\section{螺旋波的产生}
	如果在行波产生的过程中,将其突然“抹去”一半,则会导致行波传播的不均匀,从而导致了螺旋波的产生。
	
	下图为螺旋波的产生过程,其中$\varepsilon=0.04$:图\ref{t2}为螺旋波从0.05ms至720ms时的演化过程的六副截图。
	\begin{figure}[H]%H表示优先放置的图片位置
		\centering
		
		\subfigure[0.05ms]{
			\begin{minipage}[t]{0.4\linewidth}
				\centering
				\includegraphics[width=2in]{t2_0005.png}
				\label{t2_0.05}
				%\caption{fig1}
			\end{minipage}%
		}%
		\subfigure[5ms]{
			\begin{minipage}[t]{0.4\linewidth}
				\centering
				\includegraphics[width=2in]{t2_005.png}
				\label{t2_005}
				%\caption{fig2}
			\end{minipage}%
		}%	%这个回车键很重要 \quad也可以
		\quad
		\subfigure[10ms]{
			\begin{minipage}[t]{0.4\linewidth}
				\centering
				\includegraphics[width=2in]{t2_010.png}
				\label{t2_010}
				%\caption{fig2}
			\end{minipage}
		}%
		\subfigure[20ms]{
			\begin{minipage}[t]{0.4\linewidth}
				\centering
				\includegraphics[width=2in]{t2_020.png}
				\label{t2_020}
				%\caption{fig2}
			\end{minipage}
		}%
		\quad	
		\subfigure[50ms]{
			\begin{minipage}[t]{0.4\linewidth}
				\centering
				\includegraphics[width=2in]{t2_050.png}
				\label{t2_050}
				%\caption{fig2}
			\end{minipage}
		}%
		\subfigure[720ms]{
			\begin{minipage}[t]{0.4\linewidth}
				\centering
				\includegraphics[width=2in]{t2_720.png}
				\label{t2_720}
				%\caption{fig2}
			\end{minipage}
		}%
		\centering
		\caption{不同时间下的螺旋波的形态演化}
		\label{t2}
	\end{figure}


	通过调整抹去上半部分或下半部分,我们可得到回旋方向不同的螺旋波,图\ref{t2_o_020}为抹去下半部分演化至20ms的螺旋波;图\ref{t2_n_020}为抹去上半部分演化至20ms的螺旋波。
	\begin{figure}[H]%H表示优先放置的图片位置
		\centering
		
		\subfigure[抹去下半部分演化至20ms的螺旋波]{
			\begin{minipage}[t]{0.4\linewidth}
				\centering
				\includegraphics[width=2in]{t2_o_020.png}
				\label{t2_o_020}
				%\caption{fig1}
			\end{minipage}%
		}%
		\subfigure[抹去上半部分演化至20ms的螺旋波]{
			\begin{minipage}[t]{0.4\linewidth}
				\centering
				\includegraphics[width=2in]{t2_020.png}
				\label{t2_n_020}
				%\caption{fig2}
			\end{minipage}%
		}%	%这个回车键很重要 \quad也可以
		\centering
		\caption{抹去不同位置产生回旋方向不同的螺旋波}
		\label{t2_2}
	\end{figure}
	
	\section{混沌状态}
	当$\varepsilon \in [0 0.07]$时,我们能得到稳定状态下的螺旋波;而当$\varepsilon >0.07$时,我们可明显观测到螺旋波失稳而产生湍流的现象。
	
	我们令$\varepsilon = 0.1$,如图\ref{t_3}所示,我们得到了在不同时间下产生时空混沌的图像。
	
	\begin{figure}[H]%H表示优先放置的图片位置
		\centering
		
		\subfigure[10ms]{
			\begin{minipage}[t]{0.4\linewidth}
				\centering
				\includegraphics[width=2in]{t3_010.png}
				\label{t3_010}
				%\caption{fig1}
			\end{minipage}%
		}%
		\subfigure[20ms]{
			\begin{minipage}[t]{0.4\linewidth}
				\centering
				\includegraphics[width=2in]{t3_020.png}
				\label{t3_020}
				%\caption{fig2}
			\end{minipage}%
		}%	%这个回车键很重要 \quad也可以
		\quad
		\subfigure[50ms]{
			\begin{minipage}[t]{0.4\linewidth}
				\centering
				\includegraphics[width=2in]{t3_050.png}
				\label{t3_050}
				%\caption{fig2}
			\end{minipage}
		}%
		\subfigure[100ms]{
			\begin{minipage}[t]{0.4\linewidth}
				\centering
				\includegraphics[width=2in]{t3_100.png}
				\label{t3_100}
				%\caption{fig2}
			\end{minipage}
		}%
		\quad	
		\subfigure[250ms]{
			\begin{minipage}[t]{0.4\linewidth}
				\centering
				\includegraphics[width=2in]{t3_250.png}
				\label{t3_250}
				%\caption{fig2}
			\end{minipage}
		}%
		\subfigure[400ms]{
			\begin{minipage}[t]{0.4\linewidth}
				\centering
				\includegraphics[width=2in]{t3_400.png}
				\label{t3_400}
				%\caption{fig2}
			\end{minipage}
		}%
		\centering
		\caption{不同时间下时空混沌的演化图像}
		\label{t3}
	\end{figure}
	
	\section{使用行波法实现螺旋波与混沌现象的控制}
		\subsection{控制原理}
		使用行波法实现螺旋波与混沌现象的控制的原理非常简单,即在螺旋波/混沌产生之后,不断利用行波冲击螺旋波,最中达到消除螺旋波/混沌,回到行波态的过程。以下为具体的过程演示,其中$\varepsilon = 0.55$:
		
		\subsection{螺旋波的控制}
		首先,我们按照上文螺旋波方法先产生一个螺旋波,在产生螺旋波之后。给左边的边界施加一个持续的刺激,即,使u周期性输出0,1,从而产生源源不断的行波。然后观察行波冲击螺旋波的结果。
		
		下图\ref{t4_1}为在不同时间下螺旋波的产生与控制过程图像。从图中可知,控制过程比螺旋波产生过程长很多:形成一个螺旋波仅需20ms,而完全消除一个螺旋波需要392.50ms。
		
		另外,$\varepsilon$的值不同,也会导致螺旋波控制时间的不同。当$\varepsilon = 0.04$时,我们发现时间演化超过1000ms时行波依然没有一点控制螺旋波的迹象。此处有待近一步的研究探索。
		
		
		\begin{figure}[H]%H表示优先放置的图片位置
			\centering
			
			\subfigure[20ms产生螺旋波]{
				\begin{minipage}[t]{0.4\linewidth}
					\centering
					\includegraphics[width=2in]{t4_1_20.png}
					\label{t4_1_20}
					%\caption{fig1}
				\end{minipage}%
			}%
			\subfigure[螺旋波产生后行波进行控制105ms]{
				\begin{minipage}[t]{0.4\linewidth}
					\centering
					\includegraphics[width=2in]{t4_1_105.png}
					\label{t4_1_105}
					%\caption{fig2}
				\end{minipage}%
			}%	%这个回车键很重要 \quad也可以
			\quad
			\subfigure[螺旋波产生后行波进行控制200ms]{
				\begin{minipage}[t]{0.4\linewidth}
					\centering
					\includegraphics[width=2in]{t4_1_200.png}
					\label{t4_1_200}
					%\caption{fig2}
				\end{minipage}
			}%
			\subfigure[螺旋波产生后行波进行控制250ms]{
				\begin{minipage}[t]{0.4\linewidth}
					\centering
					\includegraphics[width=2in]{t4_1_250.png}
					\label{t4_1_250}
					%\caption{fig2}
				\end{minipage}
			}%
			\quad	
			\subfigure[螺旋波产生后行波进行控制372ms]{
				\begin{minipage}[t]{0.4\linewidth}
					\centering
					\includegraphics[width=2in]{t4_1_372.png}
					\label{t4_1_372}
					%\caption{fig2}
				\end{minipage}
			}%
			\subfigure[螺旋波产生后行波进行控制392.50ms]{
				\begin{minipage}[t]{0.4\linewidth}
					\centering
					\includegraphics[width=2in]{t4_1_39250.png}
					\label{t4_1_392.50}
					%\caption{fig2}
				\end{minipage}
			}%
			\centering
			\caption{螺旋波的控制}
			\label{t4_1}
		\end{figure}
		\subsection{时空混沌的控制}
		时空控制的方法与螺旋波控制的方法相同,两者像区别之处仅仅在于$\varepsilon$不同:我们此处实现时空混沌的控制时给出的$\varepsilon = 0.0701$。下图\ref{t4_2}是时空混沌的产生以及控制过程中产生的图像:
		
		
		\begin{figure}[H]%H表示优先放置的图片位置
			\centering
			
			\subfigure[50ms产生时空混沌]{
				\begin{minipage}[t]{0.4\linewidth}
					\centering
					\includegraphics[width=2in]{t4_2_50.png}
					\label{t4_2_50}
					%\caption{fig1}
				\end{minipage}%
			}%
			\subfigure[时空混沌产生后行波进行控制65ms]{
				\begin{minipage}[t]{0.4\linewidth}
					\centering
					\includegraphics[width=2in]{t4_2_65.png}
					\label{t4_2_65}
					%\caption{fig2}
				\end{minipage}%
			}%	%这个回车键很重要 \quad也可以
			\quad
			\subfigure[时空混沌产生后行波进行控制100ms]{
				\begin{minipage}[t]{0.4\linewidth}
					\centering
					\includegraphics[width=2in]{t4_2_100.png}
					\label{t4_2_100}
					%\caption{fig2}
				\end{minipage}
			}%
			\subfigure[时空混沌产生后行波进行控制165ms]{
				\begin{minipage}[t]{0.4\linewidth}
					\centering
					\includegraphics[width=2in]{t4_2_165.png}
					\label{t4_2_165}
					%\caption{fig2}
				\end{minipage}
			}%
			\quad	
			\subfigure[时空混沌产生后行波进行控制215ms]{
				\begin{minipage}[t]{0.4\linewidth}
					\centering
					\includegraphics[width=2in]{t4_2_215.png}
					\label{t4_2_215}
					%\caption{fig2}
				\end{minipage}
			}%
			\centering
			\caption{时空混沌的控制}
			\label{t4_2}
		\end{figure}
	
		由图像演化过程可知:在$\varepsilon = 0.701$下产生的时空混沌被行波完全控制所需的时间比$\varepsilon=0.55$的螺旋波被行波完全控制所需的时间要短;而且其行波控制时空混沌所需的时间仅仅需要215ms,约为2s,时间非常短,可知用行波对心脏病人进行控制治疗,从而使时空混沌消失前景十分广阔。
		
	\section{总结}
	本文基于FitzHugh-Nagumo模型,对螺旋波及其湍流态的产生以及控制做了相关的计算机模拟。在模拟过程中,我们发现使用行波对于时空混沌进行控制所需时间极短,且其操作并不复杂,便于实现。如将其运用于心脏病人的心率恢复或者其他医学领域,可能会有意想不到的效果,具有非常广阔的应用前景。
	
	\begin{appendix}%附录(代码内容)
		\section{本次课题所需用到的所有的函数程序}
		\begin{lstlisting}	
%扩散方程的非齐次项
function y = f1(u,v)
global a b ep
y = -1/ep.*u.*(u-1).*(u-(v+b)./a);
end

%f(u)函数
function y=f2(u)
global n1 n2
for i=1:n1
for j=1:n2
if u(i,j)>=0 && u(i,j)<1/3
y(i,j)=0;
elseif u(i,j)>=1/3 && u(i,j)<=1
y(i,j)=1-6.75*u(i,j)*(u(i,j)-1)^2;
elseif u(i,j)>1
y(i,j)=1;
end
end
end

%拉普拉斯算子对u作用
function la=laplace(u)
global n1 n2
%Euler法求u%
for i=2:n1-1
for j=2:n2-1
la(i,j)=u(i+1,j)+u(i-1,j)+u(i,j+1)+u(i,j-1)-4*u(i,j);
end
end
%u值的初始条件%
la(1,1)=2.*u(1,2)+2.*u(2,1)-4.*u(1,1);
la(1,n2)=2*u(1,n2-1)+2*u(2,n2)-4*u(1,n2);
la(n1,1)=2*u(n1,2)+2*u(n1-1,1)-4*u(n1,n2);
la(n1,n2)=2*u(n1,n2-1)+2*u(n1-1,n2)-4*u(n1,n2);

for j=2:n2-1
la(1,j)=u(1,j-1)+u(1,j+1)+2*u(2,j)-4*u(1,j);
la(n1,j)=u(n1,j-1)+u(n1,j+1)+2*u(n1-1,j)-4*u(n1,j);
end

for i=2:n1-1
la(i,1)=u(i-1,1)+u(i+1,1)+2*u(i,2)-4*u(i,1);
la(i,n2)=u(i-1,n2)+u(i+1,n2)+2*u(i,n2-1)-4*u(i,n2);
end
end	
		\end{lstlisting}
		\section{行波代码}
		\begin{lstlisting}	
clear all;clc;
%参数数值%
global n1 n2 la a b ep
a=0.84;
b=0.07;
n1=800;%x格点数
n2=800;%y格点数
ep=0.04;
la=zeros(n1,n2);
%拉普拉斯算符对u作用后的离散值
u=zeros(n1,n2);
%u函数离散值(初始)
v=zeros(n1,n2);
%v函数离散值(初始)
t=0;%时间初始值
dt=0.05;%时间间隔
N=60000;%时间点数
h=0.5;%相邻格点间隔

%循环遍历求得u的数值解并画图%
for i=1:N%时间循环步进
t=t+dt;
u=u+dt.*(f1(u,v)+laplace(u)/h^2);
v=v+dt.*(f2(u)-v);
if mod(t,4)>1
u(:,1)=1;
else
u(:,1)=0;
end
%imagesc(u);
imshow(u);
end
		
		\end{lstlisting}
		\section{靶波代码}
		\begin{lstlisting}	
clear all;clc;
%参数数值%
global n1 n2 la a b ep
a=0.84;
b=0.07;
n1=800;%x格点数
n2=800;%y格点数
ep=0.04;
la=zeros(n1,n2);
%拉普拉斯算符对u作用后的离散值
u=zeros(n1,n2);
%u函数离散值(初始)
v=zeros(n1,n2);
%v函数离散值(初始)
t=0;%时间初始值
dt=0.05;%时间间隔
N=60000;%时间点数
h=0.5;%相邻格点间隔

%循环遍历求得u的数值解并画图%


for i=1:N%时间循环步进
t=t+dt;
u=u+dt.*(f1(u,v)+laplace(u)/h^2);
v=v+dt.*(f2(u)-v);
if mod(t,3)<1
u(399,399) = 1;
u(399,401) = 1;
u(401,399) = 1;
u(401,401) = 1;
v(399,399) = 1;
v(399,401) = 1;
v(401,399) = 1;
v(401,401) = 1; 
else
u(399,399) = 0;
u(399,401) = 0;
u(401,399) = 0;
u(401,401) = 0;
v(399,399) = 0;
v(399,401) = 0;
v(401,399) = 0;
v(401,401) = 0;
end
if t==100
m = 1;
end
%imshow(u);%画图
imshow(u);
%xlabel('x');ylabel('y');
end	
		
		\end{lstlisting}
		\section{产生螺旋波的代码}
		\begin{lstlisting}	
clear all;clc;

%参数数值%

global n1 n2 la a b ep
a=0.84;
b=0.07;
n1=800;%x格点数
n2=800;%y格点数
ep=0.04;
la=zeros(n1,n2);
%拉普拉斯算符对u作用后的离散值
u=zeros(n1,n2);
%u函数离散值(初始)
v=zeros(n1,n2);
%v函数离散值(初始)
t=0;%时间初始值
dt=0.05;%时间间隔
N=60000;%时间点数
h=0.5;%相邻格点间隔
%螺旋波初条件%
u(401:800,398:402)=1;
v(401:800,394:398)=1;
for i=1:N
t=t+dt
u=u+dt.*(f1(u,v)+laplace(u)/h^2);
v=v+dt.*(f2(u)-v);
imshow(u);
end
		
		\end{lstlisting}
		\section{时空混沌的产生代码}
		\begin{lstlisting}	
clear all;clc;
%参数数值%
global n1 n2 la a b ep
a=0.84;
b=0.07;
n1=800;%x格点数
n2=800;%y格点数
ep=0.10;
la=zeros(n1,n2);
%拉普拉斯算符对u作用后的离散值
u=zeros(n1,n2);
%u函数离散值
v=zeros(n1,n2);
%v函数离散值
t=0;%时间初始值
dt=0.05;%时间间隔
N=60000;%时间点数
h=0.5;%相邻格点间隔

%螺旋波初条件%
u(1:400,398:402)=1;
v(1:400,394:398)=1;

%循环遍历求得u的数值解并画图%
for i=1:N%时间循环步进
t=t+dt
u=u+dt.*(f1(u,v)+laplace(u)/h^2);
v=v+dt.*(f2(u)-v);
imshow(u);
end


		
		\end{lstlisting}
		\section{控制螺旋波}
		\begin{lstlisting}	
clear all;clc;
%参数数值%
global n1 n2 la a b ep
a=0.84;
b=0.07;
n1=100;%x格点数
n2=100;%y格点数
ep=0.055;
la=zeros(n1,n2);
%拉普拉斯算符对u作用后的离散值
u=zeros(n1,n2);
%u函数离散值
v=zeros(n1,n2);
%v函数离散值
t=0;%时间初始值
dt=0.05;%时间间隔
N=60000;%时间点数
h=0.5;%相邻格点间隔

%螺旋波初条件%
u(1:50,48:52)=1;
v(1:50,44:48)=1;

%循环遍历求得u的数值解并画图%
for i=1:400%时间循环步进
t=t+dt
u=u+dt.*(f1(u,v)+laplace(u)/h^2);
v=v+dt.*(f2(u)-v);
%imshow(u)
contourf(u),pause(0.05);
end
t = 0;

for i=1:N%时间循环步进
t=t+dt
u=u+dt.*(f1(u,v)+laplace(u)/h^2);
v=v+dt.*(f2(u)-v);
if mod(t,4)>1
u(:,1)=1;
else
u(:,1)=0;
end
contourf(u),pause(0.05);
end


		
		\end{lstlisting}
		\section{段落}
		\begin{lstlisting}	
clear all;clc;
%参数数值%
global n1 n2 la a b ep
a=0.84;
b=0.07;
n1=100;%x格点数
n2=100;%y格点数
ep=0.0701;
la=zeros(n1,n2);
%拉普拉斯算符对u作用后的离散值
u=zeros(n1,n2);
%u函数离散值
v=zeros(n1,n2);
%v函数离散值
t=0;%时间初始值
dt=0.05;%时间间隔
N=60000;%时间点数
h=0.5;%相邻格点间隔

%%螺旋波初条件%
u(1:50,48:52)=1;
v(1:50,44:48)=1;

%循环遍历求得u的数值解并画图%
for i=1:1000%时间循环步进
t=t+dt
u=u+dt.*(f1(u,v)+laplace(u)/h^2);
v=v+dt.*(f2(u)-v);
%imshow(u)
contourf(u),pause(0.05);
end
t = 0;

for i=1:N%时间循环步进
t=t+dt
u=u+dt.*(f1(u,v)+laplace(u)/h^2);
v=v+dt.*(f2(u)-v);
if mod(t,4)>1
u(:,1)=1;
else
u(:,1)=0;
end
%plot(u);
%imshow(u);
contourf(u),pause(0.05);
end


		
		\end{lstlisting}
	\end{appendix}
	\begin{thebibliography}{1}
		\bibitem{b1}吴信谊 Morris-Lecar神经元放电行为及螺旋波形成机制的研究[D].兰州:兰州理工大学,2013
	\end{thebibliography}
		
\end{document}	